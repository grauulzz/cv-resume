\documentclass[singlesided,
                             paper=a4,
                             fontsize=10pt
                            ]{my-resume}


%%%%%%%%%%%%%%%%%%%%%%%%%%%%%%%%%%%%%%%%%%%%%%%%%%%%%%%%%%%%%%%%%%%%%%%%%%%%%%%%
% set geometry
%%%%%%%%%%%%%%%%%%%%%%%%%%%%%%%%%%%%%%%%%%%%%%%%%%%%%%%%%%%%%%%%%%%%%%%%%%%%%%%%

\setlength\highlightwidth{8cm}
\setlength\headerheight{3cm}            % note that margintop gets added to this value, i.e. the header bar is 5cm
\setlength\marginleft{1cm}
\setlength\marginright{\marginleft}      % needs to be 1.5 times to be actually equal. why?
\setlength\margintop{1cm}
\setlength\marginbottom{1cm}



%%%%%%%%%%%%%%%%%%%%%%%%%%%%%%%%%%%%%%%%%%%%%%%%%%%%%%%%%%%%%%%%%%%%%%%%%%%%%%%%
% FONTS
%%%%%%%%%%%%%%%%%%%%%%%%%%%%%%%%%%%%%%%%%%%%%%%%%%%%%%%%%%%%%%%%%%%%%%%%%%%%%%%%

\RequirePackage{fontspec}
\setmainfont{Carlito}


%%%%%%%%%%%%%%%%%%%%%%%%%%%%%%%%%%%%%%%%%%%%%%%%%%%%%%%%%%%%%%%%%%%%%%%%%%%%%%%%
% COLORS
%%%%%%%%%%%%%%%%%%%%%%%%%%%%%%%%%%%%%%%%%%%%%%%%%%%%%%%%%%%%%%%%%%%%%%%%%%%%%%%%

\colorlet{highlightbarcolor}{lightgray}
\colorlet{headerbarcolor}{darkgray}


\colorlet{headerfontcolor}{white}
\colorlet{accent}{awesome-red}
\colorlet{heading}{black}
\colorlet{emphasis}{black}
\colorlet{body}{black}


%%%%%%%%%%%%%%%%%%%%%%%%%%%%%%%%%%%%%%%%%%%%%%%%%%%%%%%%%%%%%%%%%%%%%%%%%%%%%%%%
% set document
%%%%%%%%%%%%%%%%%%%%%%%%%%%%%%%%%%%%%%%%%%%%%%%%%%%%%%%%%%%%%%%%%%%%%%%%%%%%%%%%


\begin{document}

\name{Will Reynolds}
\tagline{Software Engineer}
\photo[round]{IMG_20200811_131822(2)}{\dimexpr \headerheight-\marginbottom}   % make photo exactly match the header with margintop/marginright/marginbottom as margin
\makeheader


\highlightbar{

        \section{Contact}
        
        \email{billiamreynolds@gmail.com} % FIXME: email link
        \phone{+512 214 7095}
        \location{San Marcos, TX 78666}
        \vspace{0.5em}
        \github{@WillReynolds}{https://github.com/grauulzz}
        % \linkedin{Nico Krieger}{https://www.linkedin.com/in/nico-krieger-6b28151b2/}
        
        \section{Skills}
        
        \skillsection{Programming}
        \skill{Java}{5}
        \skill{Scala}{4}
        \skill{HTML/CSS}{2}
        \skill{LaTeX}{1}
        \skill{TypeScript}{4}
        \skill{C\#}{3}
        
        \vspace{0.5em}
        \skillsection{Build Frameworks}
        \skill{Gradle}{5}
        \skill{Sbt}{4}
        \skill{Angular}{3}
        
        \vspace{0.5em}
        \skillsection{Software \& Tools}
        \skill{Git}{5}
        \skill{AwsSdk}{5}
        (dynamodb, s3, lambda)\\
        \skill{WSL}{3}
        
        \vspace{0.5em}
        \skillsection{Other skills}
        (ScalaCollider, Junit, RestAPIs, Amplify)\\
        
}
\mainbar{
        \section{About Me}
        Currently, I am enrolled as a student at BloomTech and working towards a\\ career in software development.\\
        Before I decided to formalize my training, I learned the fundamentals of\\ programming on my own through the great power of google searches.\\
        Over the last couple of years, I've developed a strong appreciation\\ for the complexity and depth of computer science.


        \section[\faGears]{Work history}
        \job{06/2011 - 06/2014} 
                {TradeStar\\ Austin, Tx}
                {Entry Level Low Voltage Technician}
                {(Installed server hardware and fire alarms)}

        \job{05/2015 - 09/2017}
                {JM Electronic Engineering, Inc.\\ Pflugerville, Tx}
                {Low Voltage Technician}
                {(Installed and configured low voltage\\ network infrastructure and control panels)}

        \job{11/2017 - 02/2019} 
                {Hays Rustic Design,\\Buda, tx}
                {Woodworking}
                {(Designed and repaired various wooden structures)}
        
        \section[\faMortarBoard]{Education}
        \job{07/2011}
        {San Marcos High School}
        {Class of 2011}
        {}
        \job{11/2020 - 07/2021}
        {Self-Taught Software Development}
        {}
        {}
        \job{07/2021 - Present}
                {Bloom Institute of Technology}
                {Backend Developer Program}
                {}

        \section{Personal Traits}
        \smallskip % additional skip because tag outlines use up space
        \tag{Fast Learner}
        \tag{Low maintenance}
        \tag{Responsive to change}
        \tag{Objective}
        \tag{Consistent}
        \tag{Enthusiastic about learning}
        \tag{Creative}
        \tag{Easy to work with}
        
        \section{Computer Science Interests}
        % This is taken from AltaCV
        % see https://github.com/liantze/AltaCV for details
        \wheelchart{1.5cm}{0.5cm}{% outer and inner diameter
        6/8em/accent!20/Threat Analaysis,          % comma-separated list of
        8/8em/accent!40/Machine Learning,    % fraction of 24 / line length / color / label
        2/8em/accent!80/Digital Signal Processing,          % here, the color is shades of the accent color
        3/8em/accent!60/Game Engine Architecture,
        5/8em/accent/Computer Architecture
        }
}
\makebody
\clearpage


\pagestyle{highlightmain}

% The highlightbar needs to be filled to display mainbar contents correctly in singlesised mode
% For an empty highlightbar, fill with empty space
\highlightbar{\hfill}
\mainbar{

        \section{Another section}
        
        This page uses the page style \texttt{highlightmain} which shows the highlight bar (gray) and the main part (white background) but omits the header. 
        The default page style is \texttt{headerhighlightmain} with all three elements.
        If you don't want header, nor highlight bar, use page style \texttt{\textbackslash pagestyle\{empty\}}.
        \medskip
        Neither main, nor highlight bar must be filled to make this template work.
        It is possible to use a page style with the highlight bar but leave it empty by setting an empty highlightbar \texttt{\textbackslash highlightbar\{\}}.

        \vspace{0.5em}
        \subsection{Subsection 1}
        Demonstrate subsections.
        
        \subsection{Subsection 2}
        Subsection are also bold face but a smaller font then section. They also omit the rule.
        

}
\end{document}
